%----------------------------------------------------------------------------------------
%   
%   SUPER COOL RESUME TEMPLATE
%
%   Resume Template by Spencer Elkington, for the
%   purposes of the U Professional Education Program.
%
%   Welcome to your new resume! I'll be annotating how to apply the concepts
%   from our resume workshop below.
%   
%   Feel free to delete this top bit if you ever post the source online - take
%   credit for all of this amazing work, if you really want!
%
%   SOME BASICS:
%   
%   Anything with a '%' (percent sign) before it is COMMENTED CODE. This will keep
%   it from appearing in the final LaTeX render. I use it for two reasons:
%
%     1. Making comments like these.
%     2. Omitting things from my resume that I don't want NOW, but may want LATER.
%
%   Things will come and go from your resume - some experiences you have may not
%   feel applicable to a job you're currently applying for - DO NOT DELETE THEM.
%   You never know when experience may become relevant again!
%
%   COMMENT IT OUT USING '%', because then you can whip out your experience again
%   when you need it!
%
%   Also, sometimes it's a cool confidence booster to un-comment everything and see
%   all the neat stuff you've done so far in your professional life :)
%
%   You can find this and other resume-related code at https://github.com/Spelkington/latex-sample-resume
%
%   If you'd like to host this resume on your website, check out https://spelkington.github.io/resume-ci-pipeline/
%
%----------------------------------------------------------------------------------------


%----------------------------------------------------------------------------------------
%
%  PACKAGES AND OTHER DOCUMENT CONFIGURATIONS
%
%   These are some document-wide settings and extra LaTeX packages that you may
%   want to use for your resume. Here, you can adjust:
%     -   Margin Sizes
%     -   Indentation
%     -   Hyperlink Appearance
%
%   You will also use this section to put your NAME and CONTACT INFORMATION. This
%   stuff is very very important and I would highly recommend keeping the format
%   fairly similar. This is based on advice I've received from our alumni and other
%   job recruiters.
%
%   A NOTE ON FORMATTING: Your resume should NEVER EVER EVER be more than ONE PAGE.
%             
%       Muliple pages is best left to a CV, and unless you are a highly-qualified person
%       (read: decades of experience in your field) you really should leave it to one page.
%
%----------------------------------------------------------------------------------------

% Use the custom resume.cls style
\documentclass{./resume} 

% Document margins
\usepackage[left=0.75in,  top=0.4in,  right=0.75in,   bottom=0.3in]{geometry} 

% Stuff for setting default indentation for new paragraphs
\newcommand{\tab}[1]{\hspace{.2667\textwidth}\rlap{#1}}
\newcommand{\itab}[1]{\hspace{0em}\rlap{#1}}

% Package for proper item listing
\usepackage{enumitem}

% Package for... changing pages? I'm not too certain, and I'm not going to look it up right now.
\usepackage{changepage}

% Package for another thing I don't know about and will not look up right now.
\usepackage{parskip}

% Package for multiple columns! I'm not even sure what this is being used for. :)
\usepackage{multicol}

% Package to use *fancy* accented letters!
\usepackage[utf8]{inputenc}
\usepackage[T1]{fontenc}

%-----------------------------------------
%
%   NOTE:
%   
%   I personally like having hyperlinks to the neat stuff that I do in my resume! Especially
%   if you do not have a portfolio website, it's a fun way to directly connect digital viewers
%   of your resume to the stuff that you've done.
%
%   NOT EVERYBODY FEELS THIS WAY. It is up to your discretion whether or not you use
%   hyperlinks.
%
%   My advice, though:
%     - Keep hyperlinks set to a dark, neutral color. This will make them non-abrasive
%     compared to the black of the rest of your text, AND will make them appear normal
%     should anybody print out your resume.
%
%-----------------------------------------
\usepackage[pdfnewwindow]{hyperref}
\hypersetup{
  colorlinks=true,  % Flag for whether links should be a different color
  linkcolor=blue,   % Color of in-page links (not used)
  filecolor=magenta,  % Color of local filesystem links (not used)
  urlcolor=[rgb]{0,0,1},    % Color of web links (DEFINITELY USED!)
}

%------------------------------------------
%
%   CONTACT INFO:
%
%   DEFINITELY INCLUDE:
%     - Name
%     - Email
%     - Phone #
%
%   PROBABLY INCLUDE:
%     - ZIP Code
%     - City
%     - LinkedIn
%     - Personal Website (such as GitHub, or an actual portfolio website
%               for all of you tryhards out there.)
%
%------------------------------------------
\name{Dr. Emmit "Doc" Brown} 
\address{
  % NOTE: I include some fancy stuff in the link to my email that will automatically fill certain fields. CHANGE THIS.
  %
  % ORDER FOR LINKS: \href{ACTUAL LINK}{TEXT FOR LINK}. I choose to make the text the same as the URL, minus HTTPS://
  %
  \href{mailto:whatplutonium+resume@gmail.com?subject=Regarding\%20a\%20potential\%20work\%20opportunity&body=Hi,\%20Doc!}
     {Email} \\ 
  \href{https://www.imdb.com/name/nm0000502}{Website} \\
  \href{https://www.imdb.com/name/nm0000502}{LinkedIn} \\
  \href{https://github.com/whatplutonium}{GitHub}
}  
\address{Hill Valley, CA}

\begin{document}

%---------------------------------------------------------------------------------------
%
%  SUMMARY
%
%   If you are so inclined, you can put a summary for yourself. MAKE IT BRIEF, like
%   the tagline of a movie. A longer summary is GREAT for your LinkedIn, but the space
%   on a resume is usually better suited for things like education and experiences.
%
%   Depending on when I want it, I will comment/uncomment this section.
%
%----------------------------------------------------------------------------------------

\begin{center}
 \vspace{-1em}
 {\em
%  Engineering solutions for market and organizational friction points with reliable design and automation.
%  Uncovering stories about markets and organizations with advanced analysis and intuitive explanations.
%  Finding exciting trends \& insights in data to solve
 }
 \vspace{-10pt}
\end{center}

%----------------------------------------------------------------------------------------
%
%  EDUCATION SECTION
%
%   For a student, this should be the FIRST SECTION on your resume! It's the
%   signal that lets people know - you're going to be graduated candidate!
%
%   DEFINITELY HAVE:
%     -   Your University
%     -   Your Major(s), Minors, and Certificates (if applicable)
%     -   Your graduation date. If you do not know, USE YOUR EXPECTED GRADUATION DATE
%
%----------------------------------------------------------------------------------------

\begin{rSection}{Education}

% TODO: Edit resume.cls to handle formatting better.
{\bf University of California: Berkeley} \hfill {\em August 1953}
\vspace{2pt}
  \\ Ph.D. Theoretical Physics
  
{\bf University of Colombia} \hfill {\em August 1948}
\vspace{2pt}
  \\ B.S. Nuclear Engineering
  \\ Minors: Automobile Design | Arabic

%----------------------------------------------------------------------------------------
%
%   RELEVANT TOPICS
%
%   The following is subject to debate!
%
%   I, the mighty and benevolent creator of this template, like to put my technical
%   skills under the Education section. As a student, it feels applicable.
%
%   It is well within reason to split this into a separate skills section. People
%   recommend it all the time. Just for them, here's two magic line of code to make it
%   happen:
%
%     \end{rSection}            % End education
%     \begin{rSection}{Technical Skills}  % Start Technical Skills
%
%   There! Magical!
%   
%   This section is, if we're being honest, mostly for the benefit of the bots. Recruiters will
%   sometimes use an Applicant Tracking System (ATS) to organize resumes. These systems are
%   built to compare your resume to sets of keywords and, if you don't match the keywords
%   well enough, then into the digital abyss your resume goes.
%
%   The system below is for your benefit! It allows you to organize and hot-swap keywords
%   that apply to you. Uncomment the keywords that apply best to the job you are applying
%   for.
%
%   To verify you've covered the right keywords, I highly recommend using jobscan.co.
%   It allows you to check your resume against the job description, to make sure you
%   have as many of the keywords the ATS may be looking for as possible.
%
%   BE AWARE: WHEN YOU ARE COMMENTING/UNCOMMENTING ITEMS, DO NOT LEAVE TRAILING COMMAS.
%
%----------------------------------------------------------------------------------------

%------------------------------------------
%
%   SKILLS/RELEVANT COURSEWORK
%     - Class topics (DO NOT PUT CLASS CODES HERE.)
%     - Independent study subjects
%     - Soft skills that you happen to be good at.
%
%------------------------------------------
{\bf Key Skills:}
\vspace{-1.83em}

\begin{adjustwidth}{6em}{0pt}
  Theoretical Physics |
  Time Travel |
  Nuclear Material Management |
  Student Mentoring
  
\end{adjustwidth}


%----------------------------------------------------------------------------------------
%
%  TECHNOLOGIES
%
%   This is a place to list of software/hardware that you are proficient in. Stuff like
%   a preferred OS, software suites like Adobe, Office, or AutoDesk, common programs
%   like Git, development platforms like Unity, Unreal, Qt, etc.
%
%----------------------------------------------------------------------------------------
\vspace{-3pt}
{\bf Software:}
\vspace{-1.83em}
\begin{adjustwidth}{6em}{0pt}
  AutoDesk CAD |
  Delorean OS |
  Windows 1

\end{adjustwidth}

\end{rSection}

%----------------------------------------------------------------------------------------
%
%  LANGUAGES
%   
%   Section for programming languages.
%
%   Do NOT put a language here unless you would be comfortable white-boarding
%   or performing basic coding challenges in that language!
%
%   Additionally, I like to denote languages I am most comfortable with
%   by adding a (preferred) tag next to it.
%
%----------------------------------------------------------------------------------------
\vspace{-0.4em}
{\bf Languages:}
\vspace{-1.83em}
\begin{adjustwidth}{6em}{0pt}
  C++ |
  C Language |
  MatLab |
  Punch Cards
\end{adjustwidth}

\begin{rSection}{Experience}

  {\bf Theoretical Physics Consultant} | {\em Doc Brown's Garage \hfill June 1953 - October 2015}
  \vspace{-6pt}
  \begin{itemize}[nosep]
    \item Lead development of time travel devices, resulting in \href{https://simple.wikipedia.org/wiki/Back_to_the_Future_(franchise)}{the ability to travel back and forth through time}
    \item Managed and executed a budget of \$14 million dollars gained from an unexplained family fortune
    \item Oversaw QA testing for time travel devices, minimizing risk of maternal time-travel related incidents
  \end{itemize}
  
  {\bf Teaching Assistant} | {\em University of Colombia, Wernher von Braun Lab \hfill October 1949 - June 1953}
  \vspace{-6pt}
  \begin{itemize}[nosep]
    \item Assisted in designing physics course structure and assignments in English, Spanish and German
    \item Designed confidential rocket designs used in NASA Space Race initiatives and the Apollo Program
    \item Developed and executed university DEI initiatives and onboarding programs for transfer professors
  \end{itemize}
  
\end{rSection}

\begin{rSection}{Projects}
  
  {\bf The Delorean} {\em \hfill May 1954 - April 1985}
  \vspace{-6pt}
  \begin{itemize}[nosep]
    \item Designed vehicle modifications allowing for time travel and 37\% increased cup holder capacity
    \item Ethically sourced materials from various international Colombian and Libyan providers
    \item Coordinated business relationships with potential clients and interested parties
  \end{itemize}
  
  {\bf Doc Brown's Mega Cup-O-Matic} {\em \hfill October 1949 - June 1953}
  \vspace{-6pt}
  \begin{itemize}[nosep]
    \item Filed \href{http://www.patentlyinteresting.com/june-12.html}{a patent} for a new type of car cupholder, for storing cups of nuclear material up to 1L
    \item Developed nuclear hazard procedures for high school students interested in time and nuclear physics
  \end{itemize}

\end{rSection}

\begin{rSection}{Leadership}
 
  {\bf Doc's Kidz After-School Child Care Service} {\em \hfill October 1956 - June 1985}
  \vspace{-6pt}
  \begin{itemize}[nosep]
    \item Created community initiative to entertain and teach local student(s) about the wonders of nuclear physics
    \item Provided interesting time travel research opportunities for students to add to their college applications
  \end{itemize}

\end{rSection}

\vspace{1.2em}
\begin{center}
%  {\em References available by request} \\
\end{center}

\end{document}
